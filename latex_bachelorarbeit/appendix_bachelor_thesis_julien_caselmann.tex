\newpage

\section*{Appendix}

\subsection*{R-Script for Figures \ref{fig:alu_vs_x}, \ref{fig:time_behaviour_my_model} and \ref{fig:eigenvalue_diagram_hopf}}

\begin{lstlisting}
#
# SIRS model with reinforcement
#
# Aim:
# find a Hopf bifurcation
#
# parameter
bbeta = 5    # contact rate
alpha = 0.45  # recovery rate
B     = 2    # birthrate - determines the time scale -- fixed.
	
# we calibrate the system s.t. x=1/2 is a stat states
s.star = (B+alpha)/bbeta;
i.star = (1-0.5)*B/(alpha+B)-B/bbeta;
cat("i.star = ", i.star, "\n");
	
theta1 = 1; # reinforcement parameter of pro-vax (x)
theta2 = theta1; # reinforcement parameter of anti-vax (1-x)
n1     = 10;    # zealots pro vax
n2     = 10;    # zealots anti-vax
a      = 0.1;    # influence I
c      = 200# 1;   # time scale reinforce
	
# adapt n1, n2 s.t. a*i+n1 = a*(1-i)+n2
# choose n1 minimal, s.t. we have a non-negative n2 (=0)
n1 = max(c(a*(1-2*i.star), 0));
n2 = a*(2*i.star-1)+n1;
n.star = a*i.star+n1
theta1 = (1-2*n.star)/(1+2*n.star);
theta2 = theta1;
cat("a=", a, "n1=", n1, "n2=", n2, "n.star=", n.star, 
	"theta.pich=", (1-2*n.star)/(1+2*n.star), "\n"); 
	
theo.stat.point = c(s.star, i.star, 0.5);
	
	
#init
s = 0.8;
i = 0.2;
r = 0;
x = 0.20;
	
state = c(s,i,x);   # no r-component


# rhs of ODE
rhs <- function(state){
	s = state[1]; i = state[2]; 
	x = state[3];
	s1 = -bbeta*s*i+(1-x)*B-B*s;
	i1 = bbeta*s*i - alpha*i-B*i;
	x1 = -c*x*theta2*(1-x+n2+a*(1-i)) /
		((x+n1+a*i)+theta2*(1-x+n2+a*(1-i)));
	x1 = x1+c*(1-x)*theta1*(x+n1+a*i) / 
		(theta1*(x+n1+a*i)+(1-x+n2+a*(1-i)));
			
	return(c(s1,i1,x1));
}
	
alu <- function(x, i){
	return(
		-x*theta2*(1-x+n2+a*(1-i)) /
		 ((x+n1+a*i)+theta2*(1-x+n2+a*(1-i)))
		+(1-x)*theta1*(x+n1+a*i) /
		 (theta1*(x+n1+a*i)+(1-x+n2+a*(1-i)))
	);
}
	
curve(alu(x,i.star), from=0, to=1);
curve(alu(x,i.star+0.01), from=0, to=1, add=TRUE, col="blue");
curve(alu(x,i.star+0.1), from=0, to=1, add=TRUE, col="blue");
curve(alu(x,i.star-0.01), from=0, to=1, add=TRUE, col="red");
curve(alu(x,i.star-0.1), from=0, to=1, add=TRUE, col="red");
abline(h=0);
	
simul.plot<-function(horizont){
	# simulate
	res <<- numeric();
	aver.stat = c(0,0,0);  no.aver = 0;
	h = 0.001; tt = 0; h = 0.01;
	while (tt<horizont){
		state <<- state+h*rhs(state);
		tt    = tt + h;
		res   <<- rbind(res, c(tt, state));
		if (tt>horizont/2){
			aver.stat = aver.stat + state;
			no.aver = no.aver + 1;
		}
	}
		
	plot(res[,1], res[,2], t="l", ylim=c(0,1), 
			main = paste("c", as.character(c)),
			ylab="S(black)10*I(blue)x(orange)");
		
	lines(res[,1], 10*res[,3], t="l", col="blue")
	lines(res[,1], res[,4], t="l", col="orange")
		
	state <<- state;
	aver.state <<- aver.stat/no.aver;
}
	
	
my.tol = 1e-6;
max.iter = 500000;
	
	
# simulate to find a stat state
get.stat.state <- function(init.state){
	# forward simulation until max iter, 
	# or ||rhs|| <- my.tol
		
	my.tol2 = my.tol**2;
	my.state = init.state;
	h        = 0.01;
		
	for (i in 1:max.iter){
		loc.rhs = rhs(my.state);
		if (sum(loc.rhs**2)<my.tol2){
			return(my.state);
		}
		my.state = my.state+h*loc.rhs;
	}
	cat("# get.stat.state DID NOT CONVERGE!!!\n");
	cat("# ",loc.rhs, "\n");
	cat("# get.stat.state DID NOT CONVERGE!!!\n");
		
	return(my.state);
}
	
	
get.jacobian <- function(state){
	# compute jacobian of the vector fieeld at point "state"
	#     ( (f_1)_x, (f_2)_y, (f_3)_z )
	# J = ( (f_2)_x, (f_2)_y, (f_3)_z )
	#     ( (f_3)_x, (f_2)_y, (f_3)_z )
	J = matrix(NA, 3, 3);
	hh = 1e-3;
	
	rhsp  = rhs(state+c(hh,0,0));
	rhsm  = rhs(state+c(-hh,0,0));
	grad  = (rhsp-rhsm)/(2*hh);
	J[1,] = grad;
	rhsp  = rhs(state+c(0,hh,0));
	rhsm  = rhs(state+c(0,-hh,0));
	grad  = (rhsp-rhsm)/(2*hh);
	J[2,] = grad;
	rhsp  = rhs(state+c(0,0,hh));
	rhsm  = rhs(state+c(0,0,-hh));
	grad  = (rhsp-rhsm)/(2*hh);
	J[3,] = grad;
	
	return(J);
}
	
	
##########################################################
# go
	
# simulate 	
if (1==1){   # change to (1==1) to enable
	
	nnn = 20;
	c.list = 1+49*(0:nnn)/nnn;
	c     = c.list[1];
	all.res = c();
	for(i in 1:(nnn+1)){
		c = c.list[i];
		
		state = state+c(0.0 ,0.01, 0.0)
		simul.plot(300); 
			
		J     = get.jacobian(aver.state);
		ev    = eigen(J);
		
		all.res = rbind(all.res, 
		c(c,aver.state, rhs(aver.state), ev$values));
	}
		
	
		
}
	
#generate plot	
if (1==1){
	
	l1.x = c(); l1.y = c();
	l2.x = c(); l2.y = c();
	l3.x = c(); l3.y = c();
	for (i in 1:(nnn+1)){
		l1.x = c(l1.x, Re(all.res[i,8]));  
		l1.y = c(l1.y, Im(all.res[i,8]));
		l2.x = c(l2.x, Re(all.res[i,9]));  
		l2.y = c(l2.y, Im(all.res[i,9]));
		l3.x = c(l3.x, Re(all.res[i,10])); 
		l3.y = c(l3.y, Im(all.res[i,10]));
	}
	l.x = c(l1.x, l2.x, l3.x);
	l.y = c(l1.y, l2.y, l3.y);
		
	plot(l1.x,l1.y, xlim=c(-0.05, 0.05), ylim=c(min(l.y), 
		max(l.y)), xlab="Re(lambda)", ylab="Im(lambda)");
	points(l2.x, l2.y, col="blue");
	points(l3.x, l3.y, col="green");
	abline(h=0, lty=3); abline(v=0, lty=3);
}
	
\end{lstlisting}

\subsection{R Script used for the Data Fitting}

\begin{lstlisting}
\end{lstlisting}
	Inhalt...
\end{lstlisting}
