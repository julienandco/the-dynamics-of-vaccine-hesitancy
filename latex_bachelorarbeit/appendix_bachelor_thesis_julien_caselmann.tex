\newpage

\section*{Appendix}

\subsection*{R-Script for Figures \ref{fig:alu_vs_x}, \ref{fig:time_behaviour_my_model} and \ref{fig:eigenvalue_diagram_hopf}}

\begin{lstlisting}
#
# SIRS model with reinforcement
#
# Aim:
# find a Hopf bifurcation
#
# parameter
bbeta = 5    # contact rate
alpha = 0.45  # recovery rate
B     = 2    # birthrate - determines the time scale -- fixed.
	
# we calibrate the system s.t. x=1/2 is a stat states
s.star = (B+alpha)/bbeta;
i.star = (1-0.5)*B/(alpha+B)-B/bbeta;
cat("i.star = ", i.star, "\n");
	
theta1 = 1; # reinforcement parameter of pro-vax (x)
theta2 = theta1; # reinforcement parameter of anti-vax (1-x)
n1     = 10;    # zealots pro vax
n2     = 10;    # zealots anti-vax
a      = 0.1;    # influence I
c      = 200# 1;   # time scale reinforce
	
# adapt n1, n2 s.t. a*i+n1 = a*(1-i)+n2
# choose n1 minimal, s.t. we have a non-negative n2 (=0)
n1 = max(c(a*(1-2*i.star), 0));
n2 = a*(2*i.star-1)+n1;
n.star = a*i.star+n1
theta1 = (1-2*n.star)/(1+2*n.star);
theta2 = theta1;
cat("a=", a, "n1=", n1, "n2=", n2, "n.star=", n.star, 
	"theta.pich=", (1-2*n.star)/(1+2*n.star), "\n"); 
	
theo.stat.point = c(s.star, i.star, 0.5);
	
	
#init
s = 0.8;
i = 0.2;
r = 0;
x = 0.20;
	
state = c(s,i,x);   # no r-component


# rhs of ODE
rhs <- function(state){
	s = state[1]; i = state[2]; 
	x = state[3];
	s1 = -bbeta*s*i+(1-x)*B-B*s;
	i1 = bbeta*s*i - alpha*i-B*i;
	x1 = -c*x*theta2*(1-x+n2+a*(1-i)) /
		((x+n1+a*i)+theta2*(1-x+n2+a*(1-i)));
	x1 = x1+c*(1-x)*theta1*(x+n1+a*i) / 
		(theta1*(x+n1+a*i)+(1-x+n2+a*(1-i)));
			
	return(c(s1,i1,x1));
}
	
alu <- function(x, i){
	return(
		-x*theta2*(1-x+n2+a*(1-i)) /
		 ((x+n1+a*i)+theta2*(1-x+n2+a*(1-i)))
		+(1-x)*theta1*(x+n1+a*i) /
		 (theta1*(x+n1+a*i)+(1-x+n2+a*(1-i)))
	);
}
	
curve(alu(x,i.star), from=0, to=1);
curve(alu(x,i.star+0.01), from=0, to=1, add=TRUE, col="blue");
curve(alu(x,i.star+0.1), from=0, to=1, add=TRUE, col="blue");
curve(alu(x,i.star-0.01), from=0, to=1, add=TRUE, col="red");
curve(alu(x,i.star-0.1), from=0, to=1, add=TRUE, col="red");
abline(h=0);
	
simul.plot<-function(horizont){
	# simulate
	res <<- numeric();
	aver.stat = c(0,0,0);  no.aver = 0;
	h = 0.001; tt = 0; h = 0.01;
	while (tt<horizont){
		state <<- state+h*rhs(state);
		tt    = tt + h;
		res   <<- rbind(res, c(tt, state));
		if (tt>horizont/2){
			aver.stat = aver.stat + state;
			no.aver = no.aver + 1;
		}
	}
		
	plot(res[,1], res[,2], t="l", ylim=c(0,1), 
			main = paste("c", as.character(c)),
			ylab="S(black)10*I(blue)x(orange)");
		
	lines(res[,1], 10*res[,3], t="l", col="blue")
	lines(res[,1], res[,4], t="l", col="orange")
		
	state <<- state;
	aver.state <<- aver.stat/no.aver;
}
	
	
my.tol = 1e-6;
max.iter = 500000;
	
	
# simulate to find a stat state
get.stat.state <- function(init.state){
	# forward simulation until max iter, 
	# or ||rhs|| <- my.tol
		
	my.tol2 = my.tol**2;
	my.state = init.state;
	h        = 0.01;
		
	for (i in 1:max.iter){
		loc.rhs = rhs(my.state);
		if (sum(loc.rhs**2)<my.tol2){
			return(my.state);
		}
		my.state = my.state+h*loc.rhs;
	}
	cat("# get.stat.state DID NOT CONVERGE!!!\n");
	cat("# ",loc.rhs, "\n");
	cat("# get.stat.state DID NOT CONVERGE!!!\n");
		
	return(my.state);
}
	
	
get.jacobian <- function(state){
	# compute jacobian of the vector fieeld at point "state"
	#     ( (f_1)_x, (f_2)_y, (f_3)_z )
	# J = ( (f_2)_x, (f_2)_y, (f_3)_z )
	#     ( (f_3)_x, (f_2)_y, (f_3)_z )
	J = matrix(NA, 3, 3);
	hh = 1e-3;
	
	rhsp  = rhs(state+c(hh,0,0));
	rhsm  = rhs(state+c(-hh,0,0));
	grad  = (rhsp-rhsm)/(2*hh);
	J[1,] = grad;
	rhsp  = rhs(state+c(0,hh,0));
	rhsm  = rhs(state+c(0,-hh,0));
	grad  = (rhsp-rhsm)/(2*hh);
	J[2,] = grad;
	rhsp  = rhs(state+c(0,0,hh));
	rhsm  = rhs(state+c(0,0,-hh));
	grad  = (rhsp-rhsm)/(2*hh);
	J[3,] = grad;
	
	return(J);
}
	
	
##########################################################
# go
	
# simulate 	
if (1==1){   # change to (1==1) to enable
	
	nnn = 20;
	c.list = 1+49*(0:nnn)/nnn;
	c     = c.list[1];
	all.res = c();
	for(i in 1:(nnn+1)){
		c = c.list[i];
		
		state = state+c(0.0 ,0.01, 0.0)
		simul.plot(300); 
			
		J     = get.jacobian(aver.state);
		ev    = eigen(J);
		
		all.res = rbind(all.res, 
		c(c,aver.state, rhs(aver.state), ev$values));
	}
		
	
		
}
	
#generate plot	
if (1==1){
	
	l1.x = c(); l1.y = c();
	l2.x = c(); l2.y = c();
	l3.x = c(); l3.y = c();
	for (i in 1:(nnn+1)){
		l1.x = c(l1.x, Re(all.res[i,8]));  
		l1.y = c(l1.y, Im(all.res[i,8]));
		l2.x = c(l2.x, Re(all.res[i,9]));  
		l2.y = c(l2.y, Im(all.res[i,9]));
		l3.x = c(l3.x, Re(all.res[i,10])); 
		l3.y = c(l3.y, Im(all.res[i,10]));
	}
	l.x = c(l1.x, l2.x, l3.x);
	l.y = c(l1.y, l2.y, l3.y);
		
	plot(l1.x,l1.y, xlim=c(-0.05, 0.05), ylim=c(min(l.y), 
		max(l.y)), xlab="Re(lambda)", ylab="Im(lambda)");
	points(l2.x, l2.y, col="blue");
	points(l3.x, l3.y, col="green");
	abline(h=0, lty=3); abline(v=0, lty=3);
}
	
\end{lstlisting}

\newpage

\subsection{R Scripts used for the Data Fitting and Figure \ref{fig:data_fits}}

\subsubsection{analysis\_combined\_model.R}

\begin{lstlisting}
#
# Estimate the parameters for the combined model.
#
#  (a) For each county, estimate
#     - Zealot model (beta-distrib)
#     - All reinforcement parameters equal
#     - All parameters can be chosen independently.
# (b) Compare the models by the log-likelihood-ration test
# (c) Compare empirical and theoretical distributions 
#     by the Kolmogorov-Smirnov tests.
#

post <- function(nme){
	# remove blanks
	xx = strsplit(nme, " ");
	nme1 = paste(xx[[1]], collapse="");
	cat(nme1, "\n");
	postscript(file=nme1, pointsize=32, paper="special",
	width=8, height=9, horizontal=FALSE);
}

###############
# read data
#
#################
vaccination_data = read.csv2('my_path_to_data/vaccination_data.csv', 
	header=TRUE)

###############
# read tools
#
#################
source("parameter_estimation_reinforcement.R");


######################################################
# initialisation (apply changes ONLY here)
######################################################

#initial parameters
theta_hut.init = 0.5;
psi_hut.init  = 0.5;
ksi_hut.init = 0.5;
s_hut.init   = 100; 


#switches for work todo
analyse = TRUE;
produce.table = FALSE;
produce.figures = FALSE;


######################################################
# prepare the data
######################################################

myDataEsti = c();
impfer = vaccination_data$Wert[1:400]; #last few ones are NA
myDataEsti = impfer/100; 

#remove all values below 0.5 as we esteem them to be fixed voters
myDataEsti=myDataEsti[myDataEsti>0.5];

#renormalise it, such that 0.5 -> 0 and 1 -> 1
myDataEsti = myDataEsti * 2 - 1;

incidences = vaccination_data$Inzidenz[1:400];
inc_no_NA = get_rid_of(incidences);
#get the mean value of all incidences w/o the NA ones
dummy.i = mean(inc_no_NA);

#replace all NA values in incidences by the dummy.i
incidences = replace(incidences,dummy.i);


#set the incidence that will be used to the mean of all observed values
i.param = mean(incidences);


######################################################
# data fit
######################################################


if (analyse){
	# check optima
	s_hut.max=2500;
	
	res.tab = c();
	
	
	{
		mmean = mean(myDataEsti);
		
		
		hist(myDataEsti, freq = FALSE, nclass=30, 
		main="full reinforcement",
		xlim=c(0,1));
		
		
		###################################################################
		# first run: estimate the full reinforcement model
		###################################################################
		cat("first run", "\n");
		
		#para.ref in following order: nu_hut,theta_hut,psi_hut,ksi_hut,s_hut
		para.ref = c(mean(myDataEsti), theta_hut.init, psi_hut.init,
			ksi_hut.init, s_hut.init);   # define init para
		lll.last = lll(para.ref);
		cat("lll.last: ", lll.last, "\n");
		
		unrestricted.model     = TRUE;      # we aim at the full model
		unrestrict.theta_hut   = TRUE;
		opti.cyclic(para.ref);
		
		cat(lll.last, "\n");
		para.unrest = para.last;        # store the result
		lll.unrest  = lll.last;
		theta.res   = theta_hut*s_hut;
		
		# Kolmogorov-Smirnov test
		res.ks = ks.test(myDataEsti, function(x){pReinforce(x)}); 
		
		
		###################################################################
		# second run: reinforcement model, force equal reinforcement parameters 
		###################################################################
		
		cat("second run","\n");
		para.ref = c(mean(myDataEsti), theta_hut.init, psi_hut.init,
			ksi_hut.init, s_hut.init);  # define init para
		lll.last = lll(para.ref);
		cat(lll.last, "\n");
		hist(myDataEsti, freq = FALSE, nclass=30, 
		main="reinforcement with equal reinf. params",
		xlim=c(0,1));
		
		unrestricted.model     = TRUE;      # we aim at the full model
		unrestrict.theta_hut   = FALSE;     # we want to keep equal parameters for reinforcement
		opti.cyclic(para.ref);
		cat(lll.last, "\n");
		para.halfRestrict = para.last;        # store the result
		lll.halfRestrict  = lll.last;
		theta.res   = theta_hut*s_hut;
		
		# Kolmogorov-Smirnov test
		res.halfRestrict.ks = ks.test(myDataEsti, function(x){pReinforce(x)}); 
		
		
		###################################################################
		# third run: estimate the zealot model (beta-distrib)
		###################################################################
		
		cat("third run", "\n");
		unrestricted.model     = FALSE;           # we fix all reinforcement-paras
		unrestrict.theta_hut   = FALSE;
		
		theta_hut.zealot = 0;
		
		para.ref = c(mean(myDataEsti), theta_hut.zealot, psi_hut.init,
			ksi_hut.init,s_hut.init);
		hist(myDataEsti, freq = FALSE, nclass=30, 
		main="zealot model",
		xlim=c(0,1));
		
		opti.cyclic(para.ref);
		cat("lll.last: ", lll.last, "\n");
		para.restrict = para.last;            # store result
		lll.restrict  = lll.last;
		
		# kolmogorov-smirnov-test
		res.restric.ks = ks.test(myDataEsti, function(x){pReinforce(x)});
		
		line = c("run", "pro-vaxx", 
		theta.res,
		para.unrest,
		lll.unrest,
		para.halfRestrict,
		lll.halfRestrict,
		para.restrict,
		lll.restrict, 
		pchisq(2*(lll.unrest-lll.restrict),df=2, 
		lower.tail=FALSE), 
		pchisq(2*(lll.unrest-lll.halfRestrict),df=2, 
		lower.tail=FALSE), 
		res.ks$p.value,
		res.halfRestrict.ks$p.value,
		res.restric.ks$p.value
		);
		
		res.tab = rbind(res.tab, line);
	}
	# names orient themselves ar the supplement II of the paper
	col.names = c(
	"run", "opinion", 
	"Theta1PlusTheta2.unr",  
	"nu.unr", "theta.hat.unr", "psi.unr", "ksi.unr","s.unr",
	"lll.unr",
	"nu.halfr", "theta.hat.halfr", "psi.halfr", "ksi.halfr","s.halfr",
	"lll.halfr",
	"nu.restr", "theta.hat.restr", "psi.restr", "ksi.restr","s.restr",
	"lll.restr",
	"lll.unr.rest", "lll.unr.halfr",
	"ks.unres", "ks.halfRestr", "ks.restr");
	dimnames(res.tab)[[2]] =col.names;
	
	save(file="datAnaCombinedModel.rSave", res.tab);

}



if (produce.table){
	# produce a table
	load(file="datAnaMyModel_V1.rSave");
	sink(file="datCombinedModel.tex");
	cat(dimnames(res.tab)[[2]][c(1,2,4,5,6,7,8)]); cat(" theta1 ");
	cat(" theta2 "); cat(dimnames(res.tab)[[2]][c(22,24,26)]);
	cat("\n");
	nn = dim(res.tab)[1];
	for (j in 1:nn){
		cat(res.tab[j,1], " & ", res.tab[j,2], " & ");
		cat(res.tab[j,4], " & ", res.tab[j,5], " & ");
		cat(res.tab[j,6], " & ", res.tab[j,7], " & ", res.tab[j,8], "&");
		# theta_2 = h.s*h.theta*(1-h.psi)
		cat(as.double(res.tab[j,5])*as.double(res.tab[j,8])*
			as.double(res.tab[j,6]), " & ");
		cat(as.double(res.tab[j,5])*as.double(res.tab[j,8])*
			(1-as.double(res.tab[j,6])), " & ");
		cat(as.double(res.tab[j,22]), " & ", 
		as.double(res.tab[j,24]), " & ", 
		as.double(res.tab[j,26]), 
		"\\\\\n");
	}
	
	cat("Test hat.psi=0.5 versus free model (where is the reinforcement?)\n");
	cat(as.double(res.tab[j,23]), " \n");
	
	sink();

}



if (produce.figures){
	# produce figures
	impfer = vaccination_data$Wert[1:400];
	myDataEsti = impfer/100;       
	mmean <<- mean(myDataEsti);
	
	post(paste("Combined_model",as.character(j),".eps",sep=""));
	hist(myDataEsti, freq = FALSE, 
	main=paste("Vaccinational behaviour"),
	xlim=c(0,1), xlab="amount of pro-vaxxers x", nclass=30);
	
	para.last = as.double(res.tab[j, 4:8]);
	
	lll.last = lll(para.last);
	cat(lll.last, "\n");  mmean <<- mean(myDataEsti);
	
	curve(g(x), add=TRUE,  lwd=2);
	
	para.last = as.double(res.tab[j, 16:20]);
	lll.last = lll(para.last);
	cat(lll.last, "\n");
	curve(g(x), add=TRUE,  lwd=2, lty=2);
	dev.off();
	
	
	post(paste("Combined_model_2",as.character(j),".eps",sep=""));
	hist(myDataEsti, freq = FALSE, 
	main=paste("Vaccinational behaviour"),
	xlim=c(0,1), xlab="amount of pro-vaxxers x", nclass=30);
	
	para.last = as.double(res.tab[j, 4:8]);
	
	lll.last = lll(para.last);
	cat(lll.last, "\n");  mmean <<- mean(myDataEsti);
	
	curve(g(x), add=TRUE,  lwd=2);
	
	para.last = as.double(res.tab[j, 10:14]);
	lll.last = lll(para.last);
	cat(lll.last, "\n");
	curve(g(x), add=TRUE,  lwd=2, col = "green");
	
	para.last = as.double(res.tab[j, 16:20]);
	lll.last = lll(para.last);
	cat(lll.last, "\n");
	curve(g(x), add=TRUE,  lwd=2, lty=2);
	dev.off();
}
\end{lstlisting}

\subsubsection{parameter\_estimation\_reinforcement.R}

\begin{lstlisting}

#
# Estimate the parameters for the reinforcement model
#####################################################################
#
# general functions
#
#####################################################################

#
# myDataEsti: vector with values in (0,1); the data it uses for the estimation
#

replace <- function(vector,replacement){
	for (i in 1:length(vector)){
		if (is.na(vector[i])){
			vector[i] = replacement;
		}
	}
	return (vector);
}

get_rid_of <- function(vector){
	vector_no_NA = c();
	last_index = 1;
	for (i in 1:length(vector)) {
		if (!is.na(vector[i])){
			vector_no_NA[last_index] = vector[i];
			last_index = last_index + 1;
		}
	}
	return(vector_no_NA);
}

post <- function(nme){
	# remove blanks
	xx = strsplit(nme, " ");
	nme1 = paste(xx[[1]], collapse="");
	cat(nme1, "\n");
	postscript(file=nme1, pointsize=28, paper="special",
	width=8, height=9, horizontal=FALSE);
}

##################
#
# define distribution
#
##################

s_hut.max = 2000;

f.norm <- function(){
	# an additive constant to avoid large numbers
	# in the exponent; this number is chosen in dependence
	# on the mean value of the data at hand.
	x = mmean;
	return(-1*(s_hut*theta_hut*(0.5*x**2-phi_hut*x)
	+log(x)*((1-theta_hut)*nu_hut*(1-ksi_hut)+nu_hut*ksi_hut*i.param)*s_hut
	+log(1-x)*((1-theta_hut)*(1-nu_hut)*(1-ksi_hut)+nu_hut*ksi_hut*(1-i.param))*s_hut)
	);
}

f <- function(x){
	return(
	exp(  s_hut*theta_hut*(0.5*x**2-phi_hut*x)
	+log(x)*((1-theta_hut)*nu_hut*(1-ksi_hut)+nu_hut*ksi_hut*i.param)*s_hut
	+log(1-x)*((1-theta_hut)*(1-nu_hut)*(1-ksi_hut)+nu_hut*ksi_hut*(1-i.param))*s_hut +f.norm()
	)
	);
}

get.cc <- function(){
	# get the normalisation constant
	CC=(integrate(f, lower=10**-6,upper=1-10**-6)$value)**(-1);
	return(CC);
}


g<-function(x){
	# density (note: CC is computed using the constant f.norm().
	# That constant cancels at the end of the day.
	return(
	CC*
	exp(s_hut*theta_hut*(0.5*x**2-phi_hut*x)
	+log(x)*((1-theta_hut)*nu_hut*(1-ksi_hut)+nu_hut*ksi_hut*i.param)*s_hut
	+log(1-x)*((1-theta_hut)*(1-nu_hut)*(1-ksi_hut)+nu_hut*ksi_hut*(1-i.param))*s_hut +f.norm()
	)
	);
}

lll.dat <- function(x,nu_hut,theta_hut,phi_hut,ksi_hut,s_hut, CC){
	# log likelihood for one single data point x
	return(
	s_hut*theta_hut*(0.5*x**2-phi_hut*x)
	+log(x)*((1-theta_hut)*nu_hut*(1-ksi_hut)+nu_hut*ksi_hut*i.param)*s_hut
	+log(1-x)*((1-theta_hut)*(1-nu_hut)*(1-ksi_hut)+nu_hut*ksi_hut*(1-i.param))*s_hut
	+log(CC)+f.norm()
	);
}

pReinforce.loc <- function(x) {
	# parameters given by global parameters;
	# particularly, CC is defined.
	res = integrate(g,lower = 0, upper = x);
	return(res$value);
}

pReinforce <- function(x){
	return(sapply(x, pReinforce.loc));
}

#######################
# estimate paras:
# nu_hut, theta_hut, phi_hut, ksi_hut, B
########################
theta_hut.max = 1800;
theta_hut.max = 1900;
tryCatch.W.E <- function(expr){
	W <- NULL
	w.handler <- function(w){ # warning handler
	W <<- w
	invokeRestart("muffleWarning")
	}
	list(value = withCallingHandlers(tryCatch(expr, error = function(e) e),
	warning = w.handler),
	warning = W)
}

# compute log likelihood
lll <- function(para){
	ppara     <<- para;
	nu_hut    <<- para[1];
	theta_hut <<- para[2];
	phi_hut   <<- para[3];
	ksi_hut   <<- para[4];
	s_hut     <<- min(s_hut.max,abs(para[5]));
	OK = TRUE;
	aa = tryCatch.W.E(get.cc());
	aa <<- aa;
	if (!(is.double(aa$value))>0) return(-10000);
	CC<<- aa$value;
	
	return(sum(lll.dat(myDataEsti, nu_hut,theta_hut,phi_hut,ksi_hut,s_hut,CC)));    
}


#
# for the optimization: vary only one of the parameters
#
search.p1 <- function(px){
	# para.last gives the framework; we modify parameter 1 only
	pxx = para.last; pxx[1] = px;
	return( lll(pxx) );
}
search.p2 <- function(px){
	# para.last gives the framework; we modify parameter 2 only
	pxx = para.last; pxx[2] = px;
	return( lll(pxx) );
}
search.p3 <- function(px){
	# para.last gives the framework; we modify parameter 3 only
	pxx = para.last; pxx[3] = px;
	return( lll(pxx) );
}
search.p4 <- function(px){
	# para.last gives the framework; we modify parameter 4 only
	pxx = para.last; pxx[4] = px;
	return( lll(pxx) );
}



############################################
#
# optimisation
#
############################################

opti.cyclic <- function(para.init){
	# optimise cyclically the parameters.
	#
	# we have different modes 
	# unrestricted.model == FALSE: fix all parameters expect of s_hut, and nu_hut => we can take
	#                              the reinforcement to zero and fit a beta distribution.
	# unrestrict.theta_hut   == TRUE:  allow theta_hut to vary.
	#                              (if FALSE: we can fix theta_hut=0.5, and in this way,
	#                               try to find out where the reinforcement takes place)
	
	mmean <<- mean(myDataEsti);
	
	para.last <- para.init; 
	lll.last  <- lll(para.ref);
	last.s_hut = -1; no.s_hut.const = 0;last.lll=-1e10;
	fertig = FALSE;
	
	i = 0;
	while ((i<10000)&(fertig==FALSE)){
		i = i+1;
		cat("para.last:", para.last,"\n");
		lll.x = lll.last;
		para.last <<- para.last;
		res1 = optimize(search.p1, interval=c(0,1), maximum=TRUE);
		para.loc1 = para.last; para.loc1[1]=res1$maximum;
		lll.lok  = lll(para.loc1);
		if (lll.lok>lll.last){
			para.last <- para.loc1;
			lll.last  <- lll.lok;
		}
		
		##optimize ksi
		para.last <<- para.last;
		res4 = optimize(search.p4, interval=c(0,1), maximum=TRUE);
		para.loc4 = para.last; para.loc4[4]=res4$maximum;
		lll.lok  = lll(para.loc4);
		if (lll.lok>lll.last){
			para.last <- para.loc4;
			lll.last  <- lll.lok;
		}
		
		if (unrestricted.model){
			para.last <<- para.last;
			res2 = optimize(search.p2, interval=c(0,1), maximum=TRUE);
			para.loc2 = para.last; para.loc2[2]=res2$maximum;
			lll.lok  = lll(para.loc2);
			if (lll.lok>lll.last){
				para.last <- para.loc2;
				lll.last  <- lll.lok;
			}
		
			if (unrestrict.theta_hut) {
				para.last <<- para.last;
				res3 = optimize(search.p3, interval=c(0,1), maximum=TRUE);
				para.loc3 = para.last; para.loc3[3]=res3$maximum;
				lll.lok  = lll(para.loc3);
				if (lll.lok>lll.last){
					para.last <- para.loc3;
					lll.last  <- lll.lok;
				}
			}
		}
		
		lll.1     = lll(para.last);
		Delta = 0.01;
		if (para.last[5]>10)  Delta=0.1;
		if (para.last[5]>50)  Delta=0.5;
		if (para.last[5]>100) Delta=1;
		
		lll.p2    = lll(para.last+c(0,0, 0,0, Delta));
		lll.m2    = lll(para.last+c(0,0,0,0, -Delta));
		if (lll.p2>lll.1){
			paral.loc3 = para.last+c(0,0,0,0, Delta);
			para.last  = para.last+c(0,0,0,0, Delta);
			last.lll = lll.p2;
		} else {
			if (lll.m2>lll.1){
				paral.loc3 = para.last+c(0,0,0,0,-Delta);
				para.last  <- para.last+c(0,0,0,0,-Delta);
				last.lll   <- lll.m2
			} else {
				paral.loc3 = para.last+c(0,0,0,0,0);
				para.last  <- para.last+c(0,0,0,0,0);
				last.lll   <- lll.1
			}
		}
		if (last.s_hut!=para.last[3]){
			last.s_hut = para.last[3];
			no.s_hut.const = 0;
			lll.x =last.lll;
		} else {
			no.s_hut.const = no.s_hut.const+1;
			loc.lll = lll(para.last);
			if (last.lll>lll.x+1e-6) {
				no.s_hut.const = 0;
			}
			if(no.s_hut.const>100) fertig=TRUE;
		}
		
		para.last <<- para.last;
		lll.last  <<- last.lll;
		cat(i," ", lll(para.last), "\n");
		
		if(fertig){
			curve(g(x), add=TRUE, col="red");
		}
		
	}
}
\end{lstlisting}

