	\begin{equation*}
		\Theta_X = \hat{s}\hat{\theta}\hat{\psi}\textnormal{, }\Theta_Y = \hat{s}\hat{\theta}(1-\hat{\psi})\textnormal{, }N_X - 1 = \hat{s}(1-\hat{\theta})\hat{\nu}(1-\hat{\xi})\textnormal{, }N_Y - 1 = \hat{s}(1-\hat{\theta})(1-\hat{\nu})(1-\hat{\xi})\textnormal{, }A=\hat{s}\hat{\theta}\hat{\xi},
	\end{equation*}
	where $\hat{s},\hat{\theta},\hat{\nu},\hat{\xi} \in \left[0,1\right]$ and $\hat{s} > 0$, with the restriction $\hat{s}(1-\hat{\theta})\hat{\nu}(1-\hat{\xi}) > 1$ and $\hat{s}(1-\hat{\theta})(1-\hat{\nu})(1-\hat{\xi}) > 1.$ Therewith, the distribution becomes
	\begin{align*}
		\phi(x) &= \hat{C}\exp\left[\hat{s}\Big\{\hat{\theta}\left(\frac{x^2}{2} - \hat{\psi}x\right)+\ln(x)\left((1-\hat{\theta})\hat{\nu}(1-\hat{\xi})+\hat{\theta}\hat{\xi}i\right)\right.\Big.\\
			&\qquad \Big.\left.+ \ln(1-x)\left((1-\hat{\theta})(1-\hat{\nu})(1-\hat{\xi})+\hat{\theta}\hat{\xi}(1-i)\right) + B\Big\}\right]
	\end{align*}
		
	Here, $B$ is a constant that can be chosen in dependency on the data at hand. In practice, it is used to avoid an exonent that has a very large absolute number. The constant $\hat{C}$ is, as before, determined by the fact that the integral is 1.