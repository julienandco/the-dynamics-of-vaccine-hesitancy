\documentclass[10pt,a4paper]{article}
\usepackage[english]{babel}
\usepackage{amsmath}
\usepackage{amsthm}
\usepackage{amssymb}
\usepackage{amsfonts}
\usepackage{graphicx}
\usepackage{mathtools}
\usepackage[utf8]{inputenc}
\usepackage{csquotes}
\usepackage[hidelinks, final]{hyperref}
\usepackage[printonlyused]{acronym}
\usepackage{color}
\usepackage{transparent}
\begin{document}
	\begin{center}
		\underline{Ideen für Erweiterung des Zelotenmodells m. Reinforcement}
	\end{center}
\textbf{Idee 1}\newline
Zwei Payoff-Funktionen: die eine für Impfgegner ($u_{con}$), die andere für Befürworter ($u_{pro}$). Diese Payoffs können positiv (impfen lohnt sich für mich) bzw. negativ sein (lohnt sich nicht). Danach wird auf den Payoff die Sigmoid Funktion angewandt, um aus dem Payoff eine Impfwahrscheinlichkeit zu machen. Diese zwei Wahrscheinlichkeiten werden jeweils in das Zeloten Modell m. Reinforcement übernommen (als Impfraten für die beiden jeweiligen Gruppen).
	\begin{align*}
		u_{pro}(\omega, \theta, I, \epsilon) &= [(1-1000\omega) \times I \times \epsilon \times (\theta + \lambda_{\theta}) \\
		&\qquad - (1000\omega - \lambda_{\omega})(1-\epsilon)(1-\theta)(1-I)] \times 10\\
	&\\
		u_{con}(\omega, \theta, I, \epsilon) &= (\theta - \mu_{\theta})(\epsilon - \mu_{\epsilon})(\omega + \mu_{\omega}) - \omega(1-\epsilon)\theta \times I \textbf{   TODO}
	\end{align*}
	
Hierbei sind $\lambda_{\theta}, \lambda_{\omega}, \mu_{\theta}, \mu_{\omega}$ und $\mu_{\epsilon}$ Bias-Werte. Ein Impfgegner wird die Wirksamkeit der Impfung drastisch und die Lethalität der Krankheit ein wenig abwerten wollen und dafür das Impfrisiko verstärkt wahrnehmen, wohingegen ein Befürworter das Risiko etwas kleinreden wird und die Gefahr der Krankheit verstärkt berücksichtigen.\newline
In die Transition $X \rightarrow X + 1$ kommt dann die Impfwahrscheinlichkeit eines Impfgegners $(\mathrm{sig}\left( u_{con}\right))$, denn in dieser Transition geht es ja darum einen neuen Impfer zu gewinnen und analog kommt in die Transition $X \rightarrow X-1$ die Wslkt für Impfbefürworter, nicht mehr zu impfen $(\mathrm{sig}\left( -u_{pro}\right))$.\newline
Das Minus vor $u_{pro}$ entsteht aus der Eigenschaft der Sigmoidfunktion, negative Werte auf geringe Wslkten und positive auf hohe Wslkt zu mappen. Wenn also der Payoff für einen Impfbefürworter erneut zu impfen, sehr groß ist, dann wird dieser mit sehr geringer Wslkt nicht impfen. Daher sig(-upro).(Bei Impfgegnern ist Lagerwechsel und Payoff pos. korreliert, bei Befürwortern negativ)\newline
$\Rightarrow$ Ansatz für das Modell:
\begin{align*}
\begin{split}
X_t \rightarrow X_t + 1 & \textnormal{ at rate } \mu\left(N-X_t\right)\frac{\theta_X (X_t + N_X)}{\theta_X (X_t + N_X) + (N - X_t + N_Y)} \times \mathrm{sig}\left(u_{con}(\omega, \theta, I, \epsilon)\right) \\
&\\
X_t \rightarrow X_t - 1 & \textnormal{ at rate } \mu X_t\frac{\theta_Y(N-X_t+N_Y)}{(X_t + N_X) + \theta_Y(N-X_t+N_Y)} \times \mathrm{sig}\left(- u_{pro}(\omega, \theta, I, \epsilon)\right)
\end{split}
\end{align*}
\textbf{Probleme von Idee 1:}\newline
Die Werte der Payoff-Funktion für Befürworter sehen ok aus, nur bei den Impfgegnern läuft noch was schief: sie wechseln jedes mal zu pro-impfen!!! sehr unplausibel
\newpage
\textbf{Idee 2}\newline
Wieder payoff funktionen. Diesmal eher als Counter (min/max: -4/4, damit Sigmoid Funktion anständige Wslkt ausspuckt). Genau derselbe Ansatz wie Idee 1, nur Funktionen anders:
\begin{align*}
	siehe Python Datei
\end{align*}
\textbf{TODO: payoff für con ist immer negativ $\rightarrow$ impfen wirklich nie, was auch unplausibel ist. Außerdem wurden die Funktionen mit if-else statements implementiert. Mathematisches Äquivalent dazu?}
\newpage
\textbf{Idee 3}\newline
Anmerkung in Meeting 4: persönliche Payoffberechnung und Einflussnahme anderer sollen nicht parallel voneinander passieren, sondern abhängig sein. Ansatz mit Payoff gefällt, also wird dieser beibehalten und die Funktionen werden an die Anmerkung angepasst.\newline
$\Rightarrow$ Zwei simple Payoff-Terme:
\begin{align*}
	\textnormal{spricht für Impfung:}& \theta + I\\
	\textnormal{spricht gg. Impfung:}& 1-\epsilon + 100000\omega
\end{align*}
Bei der Transition $X \rightarrow X + 1$ muss ein Impfgegner zu einem Befürworter ``konvertiert" werden. Der sog. ``influence factor" $\Pi_{pro}$ (Einfluss, den die Pro-Leute haben) wird also davon abhängen, wieviele Leute impfen und wieviele Pro-Impf-Zeloten es gibt:
\begin{equation}
	\Pi_{pro} = X_t + p_{meet Z_1}
\end{equation}
Wobei $p_{meet Z_1}$ die Wahrscheinlichkeit beschreibt, einen Pro(1)-Impf-Zeloten zu treffen (nach dem Zel-Modell m. Reinforcement):
\begin{equation}
	p_{meet Z_1} = \frac{\theta_X N_X}{\theta_X(N_X + X_t) + (N-X_t + N_Y)}.
\end{equation}
Jetzt können wir unsere neue Payoff-Funktion für Impfgegner, die Befürworter werden sollen, aufschreiben:
\begin{equation}
	u_{con}\left(\omega, \theta, I, \epsilon, \Pi_{pro}\right) = \Pi_{pro} * (\theta + I) - (2 -\Pi_{pro})(1-\epsilon + 100000\omega).
\end{equation}
Komplett analog definieren wir für die Transition $X \rightarrow X-1$ die folgenden Variablen:
\begin{align*}
	\Pi_{con} &= N-X_t + p_{meet Z_0}\\
	&\\
	p_{meet Z_0} &= \frac{\theta_Y N_Y}{X_t + N_X + \theta_Y(N-X_t+N_Y)}
\end{align*}
Jetzt können wir den Payoff für Impfbefürworter, die Gegner werden sollen, aufschreiben:
\begin{equation}
	u_{pro}\left(\omega, \theta, I, \epsilon, \Pi_{con}\right) = (2-\Pi_{con})(\theta + I) - \Pi_{con} (1-\epsilon+100000\omega)
\end{equation}
$\Rightarrow$ Modell:
\begin{align*}
\begin{split}
X_t \rightarrow X_t + 1 & \textnormal{ at rate } \mu\left(N-X_t\right)\frac{\theta_X (X_t + N_X)}{\theta_X (X_t + N_X) + (N - X_t + N_Y)} \times \mathrm{sigmoid}\left(u_{con}\right) \\
&\\
X_t \rightarrow X_t - 1 & \textnormal{ at rate } \mu X_t\frac{\theta_Y(N-X_t+N_Y)}{(X_t + N_X) + \theta_Y(N-X_t+N_Y)} \times \mathrm{sigmoid}\left(- u_{pro}\right)
\end{split}
\end{align*}
\newpage
\textbf{Idee 4}\newline
\begin{align*}
\begin{split}
X_t \rightarrow X_t + 1 & \textnormal{ at rate } \mu\left(N-X_t\right)\frac{\theta_X (X_t + N_X + \theta + I)}{\theta_X (X_t + N_X + \theta + I) + (N - X_t + N_Y + 1-\epsilon + \kappa\omega)} \\
&\\
X_t \rightarrow X_t - 1 & \textnormal{ at rate } \mu X_t\frac{\theta_Y(N-X_t+N_Y+1-\epsilon+\kappa\omega)}{(X_t + N_X+\theta+I) + \theta_Y(N-X_t+N_Y+1-\epsilon+\kappa\omega)}
\end{split}
\end{align*}
\end{document}