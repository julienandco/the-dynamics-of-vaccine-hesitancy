\documentclass[10pt,a4paper]{article}
\usepackage[english]{babel}
\usepackage{amsmath}
\delimitershortfall=-1pt
\usepackage{amsthm}
\usepackage{amssymb}
\usepackage{amsfonts}
\usepackage{graphicx}
\usepackage{mathtools}
\usepackage[utf8]{inputenc}
\usepackage{csquotes}
\usepackage[hidelinks, final]{hyperref}
\usepackage[printonlyused]{acronym}
\usepackage{color}
\usepackage{transparent}
\newtheorem{model}{Model}[section]
\begin{document}
	We know by following common $SIR$-modeling approaches with standard incidence, that our stochastic process looks like this:
	\begin{align}\label{def:my_model}
	\begin{split}
		\dot{S} &= -\beta S\frac{I}{N} + bN\left(1-\frac{X_t}{N}\right) - \mu S\\
		\dot{I} &= \beta S\frac{I}{N}  - \alpha I - \mu I\\
		\dot{R} &= \alpha I + bN\frac{X_t}{N} - \mu R\\
		\dot{N} &= bN - \mu N\\
		X_t &\rightarrow X_t \pm 1
		\end{split}
	\end{align}
	This is a stochastic process, but we want to work with a system of ODEs. To do so, we will have to compute the deterministic limit of all of the components of the model $(S,I, R and X_t)$. We define:
	\begin{equation}\label{def:scaled_variables_my_model}
		s\coloneqq \frac{S}{N}, i\coloneqq \frac{I}{N}, r\coloneqq \frac{R}{N}, x\coloneqq \frac{X_t}{N}
	\end{equation}
	We may now compute:
	\begin{align*}
\dot{s} &= \frac{N\dot{S}-S\dot{N}}{N^2} = \frac{\dot{S}}{N} - \frac{S}{N}\frac{\dot{N}}{N}\\
&\overset{(\ref{def:my_model}, \ref{def:scaled_variables_my_model})}{=} -\beta si + b\left(1-x\right) - bs\\
&\\
\dot{i} &= \frac{N\dot{I}-I\dot{N}}{N^2} = \frac{\dot{I}}{N} - \frac{I}{N}\frac{\dot{N}}{N}\\
&\overset{(\ref{def:my_model}, \ref{def:scaled_variables_my_model})}{=} \beta si - i\left(\alpha + b\right)\\
&\\
\dot{r} &= \frac{N\dot{R}-R\dot{N}}{N^2} = \frac{\dot{R}}{N} - \frac{R}{N}\frac{\dot{N}}{N}\\
&\overset{(\ref{def:my_model}, \ref{def:scaled_variables_my_model})}{=} \alpha i + bx - rb
\end{align*}
Note that $\dot{r}$ is not needed in the model anymore, since it does not occur in a single one of the other equations, giving us a three-dimensional system! Thus, our final model reads:
\begin{model}[My model V1]
	\begin{align}
		\begin{split}
			\dot{s} &= -\beta si + b\left(1-x\right) - bs\\
			\dot{i} &= \beta si - i\left(\alpha + b\right)\\
			\dot{x} &= -\mu x\frac{\theta_Y(1-x+n_Y+\Omega)}{(x+n_X+ai) + \theta_Y(1-x+n_Y+\Omega)}\\
			&\qquad+ \mu \left(1-x\right)\frac{\theta_X (x+ n_X+ ai)}{\theta_X (x + n_X + ai) + (1-x + n_Y + \Omega)}
		\end{split}
	\end{align}
\end{model}

Note: every $b$ that occured in $\dot{x}$ was replaced by $\Omega$ to ensure there is no confusion with the birth rate $b$ used in $\dot{s}$ and $\dot{i}$.
	%\newpage
	%\textbf{second try (10/05/2020)}
\begin{prop}\label{result_2}
	Let $N_X = n_X N, N_Y = n_Y N$ and $i \coloneqq \frac{I}{N}$. Then, the determinstic limit for $x\left(t\right) = \frac{X_t}{N}$ reads
	\begin{align*}
	\begin{split}
	\dot{x} = &-\mu x\frac{\theta_Y(1-x+n_Y+b)}{(x+n_X+ai) + \theta_Y(1-x+n_Y+b)}\\
	\qquad&+ \mu \left(1-x\right)\frac{\theta_X (x+ n_X+ ai)}{\theta_X (x + n_X + ai) + (1-x + n_Y + b)}
	\end{split}
	\end{align*}
\end{prop}
\begin{proof}\phantom{lol}\\
	\textbf{1. Master equation}\newline
	For better readability and improved simplicity, we will refer to the transition rates of the model as:
	\begin{align*}
	F_+ \left(X_t\right) &\coloneqq\frac{\theta_X (X_t + N_X + aI)}{\theta_X (X_t + N_X + aI) + (N-X_t + N_Y + bN)} \\
	&\\
	F_- \left(X_t\right) &\coloneqq\frac{\theta_Y(N-X_t + N_Y + bN)}{(X_t + N_X + aI) + \theta_Y(N-X_t + N_Y + bN)}
	\end{align*}
	$\Rightarrow$ the master equation reads:
	\begin{align}\label{eq:master_eq_2}
	\begin{split}
	\dot{p_k}\left(t\right) &= F_+ \left(k-1\right)\times p_{k-1}\left(t\right) + F_- \left(k+1\right)\times p_{k+1}\left(t\right)\\
		&\qquad - \left(F_+\left(k\right) + F_-\left(k\right)\right)\times p_k\left(t\right)
	\end{split}
	\end{align}
	\textbf{2. Kramer-Moyal expansion}\newline
	Now, we assume
	\begin{equation}\label{assumptions_2}
		p_k(t) \approx hu\left(\frac{k}{N}, t\right), h=\frac{1}{N}\textnormal{ and }x = hk.
	\end{equation} 
	Again, we define for simplicity by expanding $F_+(x)$ and $F_-(x)$ with $\frac{h}{h}$, respectively:
	\begin{align}\label{def:daggers_2}
	\begin{split}
	f_+(x) &\coloneqq \frac{\theta_X (x+ n_X+ ai)}{\theta_X (x + n_X + ai) + (1-x + n_Y + b)}\\
	f_-(x) &\coloneqq \frac{\theta_Y(1-x+n_Y+b)}{(x+n_X+ai) + \theta_Y(1-x+n_Y+b)}
	\end{split}
	\end{align}
	We compute:
	\begin{align}\label{computation_2}
	\begin{split}
	\partial_t \left(u\left(\frac{k}{n}, t\right)\right) &= \frac{1}{h}\dot{p_k}\left(t\right)\\
	&\overset{(\ref{eq:master_eq_2},\ref{assumptions_2})}{=} \frac{1}{h}\left[F_+(k-1)hu(x-h,t) + F_-(k+1)hu(x+h,t)\right.\\
		&\qquad\left. - \left(F_+(k)+F_-(k)\right)hu(x,t)\right]\\
	&= \frac{1}{h}\left[\textcolor{blue}{\mu(1-(x-h))\times f_+(x-h) \times u\left(x-h,t \right)}\right.\\
		&\qquad\left. + \textcolor{red}{\mu(x+h)\times f_-(x+h)\times u\left(x+h,t\right)}\right.\\
		&\qquad\left.- \left(\mu(1-x)\times f_+(x) + \mu x\times f_-(x)\right)\times u\left(x,t\right)\right] 
	\end{split}
	\end{align}
	We will now compute the Taylor series of second order for the \textcolor{blue}{blue} and \textcolor{red}{red} terms:
	\begin{align}\label{taylor_results_2}
	\begin{split}
	&\textcolor{blue}{\mu(1-(x-h))\times f_+(x-h) \times u\left(x-h,t \right)}\\
	&= \mu(1-x)f_+(x) u\left(x,t\right) - h\partial_x \left(\mu(1-x)f_+(x)u\right)\\
		&\qquad +\frac{1}{2}h^2\partial^2_x\left(\mu(1-x)f_+(x)u\right)\\
	&\\
	&\\
	&\textcolor{red}{\mu(x+h)\times f_-(x+h) \times u\left(x+h,t\right)} \\
	&= \mu x f_-(x)u(x,t) + h\partial_x \left(\mu x f_-(x) u\right) + \frac{1}{2}h^2\partial^2_x\left(\mu xf_-(x) u\right)
	\end{split}
	\end{align}
	We may now resume the computations \eqref{computation_2} from above and add those new results:
	\begin{align*}
	\partial_t \left(u\left(\frac{k}{n}, t\right)\right) &= \dots \\
	&\overset{(\ref{assumptions_2}, \ref{taylor_results_2})}{=} -\partial_x\big(\mu u(x,t)\left((1-x)f_+(x) - xf_-(x)\right)\big)\\
	&\qquad + \frac{1}{2N} \partial^2_x\left(\mu u(x,t)\left((1-x)f_+(x) + xf_-(x)\right)\right)\\
	\end{align*}
	So, for $N \rightarrow \infty$:
	\begin{equation}
		\partial_t \left(u\left(\frac{k}{n}, t\right)\right) = -\partial_x\left(\mu u(x,t)\left((1-x)f_+(x) - xf_-(x)\right)\right)
	\end{equation}

	By the definition in \eqref{def:daggers_2}, this means that the ODE due to the drift term in case $N \rightarrow \infty$ reads 
	\begin{align*}
	\dot{x} = &-\mu x\frac{\theta_Y(1-x+n_Y+b)}{(x+n_X+ai) + \theta_Y(1-x+n_Y+b)}\\
		\qquad&+ \mu \left(1-x\right)\frac{\theta_X (x+ n_X+ ai)}{\theta_X (x + n_X + ai) + (1-x + n_Y + b)}
	\end{align*}
	Which is exactly the result stated in Proposition \ref{result_2}
\end{proof}
\end{document}