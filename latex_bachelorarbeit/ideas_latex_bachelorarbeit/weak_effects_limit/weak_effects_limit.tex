\documentclass[10pt,a4paper]{article}
\usepackage[english]{babel}
\usepackage{amsmath}
\delimitershortfall=-1pt
\usepackage{amsthm}
\usepackage{amssymb}
\usepackage{amsfonts}
\usepackage{graphicx}
\usepackage{mathtools}
\usepackage[utf8]{inputenc}
\usepackage{csquotes}
\usepackage[hidelinks, final]{hyperref}
\usepackage[printonlyused]{acronym}
\usepackage{color}
\usepackage{transparent}
\newtheorem{prop}{Proposition}[section]
\begin{document}
	We take a look back at the Fokker-Planck equation for which we computed the Kramers-Moyal expansion before. %TODO: reference
	There, the scaling was $n_i = N_i/N$ and $\theta_i$ constant in N. Afterwards, we will use a slightly different scaling which we will prefer for the Dirichlet limit. We recall the transition rates:
	\begin{align}\label{def:abbreviation_scaled_transition_rates}
	\begin{split}
	f_+(x) &\coloneqq \frac{\theta_X (x+ n_X+ ai)}{\theta_X (x + n_X + ai) + (1-x + n_Y + a(1-i))}\\
	f_-(x) &\coloneqq \frac{\theta_Y(1-x+n_Y+a(1-i))}{(x+n_X+ai) + \theta_Y(1-x+n_Y+a(1-i))}
	\end{split}
	\end{align}
	
	Thus, the limiting Fokker-Planck equation reads:
	\begin{align}
	\begin{split}
		\partial_t \left(u\left(x, t\right)\right) &= -\partial_x\big(\mu u(x,t)\left((1-x)f_+(x) - xf_-(x)\right)\big) \\
		&\qquad + \frac{1}{2N} \partial^2_x\left(\mu u(x,t)\left((1-x)f_+(x) + xf_-(x)\right)\right)
	\end{split}
	\end{align}
	
By rewriting drift and noise term with $n_i = N_i/N, \theta_i = 1- \frac{\Theta_i}{N}$, $h = 1/N, a = A/N$ and neglecting terms of $\mathcal{O}(N^{-2})$ we find (using maxima %TODO: cite
):

\begin{align*}
& \mu(1-x)f_+(x) - \mu xf_-(x)\\
= & \mu(1-x) \frac{(1-h\Theta_X) (x+ hN_X+ hAi)}{(1-h\Theta_X) (x + hN_X + hAi) + (1-x + hN_Y + hA(1-i))}\\
&\qquad - \mu x \frac{(1-h\Theta_Y)(1-x+hN_Y+hA(1-i))}{(x+h_NX+hAi) + (1-h\Theta_Y)(1-x+hN_Y+hA(1-i))}\\
=& \mu\Big(\left[\left(\Theta_X + \Theta_Y\right)x - \Theta_X\right]x\left(1-x\right)-\left(N_X + N_Y + A\right)x + Ai + N_X\Big)h+ \mathcal{O}(h^2),\\
\end{align*}

while 

\begin{equation*}
h\left[\mu(1-x)f_+(x) + \mu xf_-(x)\right] = \mu h 2x(1-x) + \mathcal{O}(h^2).
\end{equation*}

After rescaling time: $T = \mu h t$, the Fokker-Planck equation becomes

\begin{align*}
\partial_T \big(u\left(x, T\right)\big) &= -\partial_x\Bigg(u(x,T)\Big(\left[\left(\Theta_X + \Theta_Y\right)x - \Theta_X\right]x\left(1-x\right)-\left(N_X + N_Y + A\right)x + Ai + N_X\Big)\Bigg)\\
&\qquad + \partial^2_x\big(u(x,T)x(1-x)\big)
\end{align*}

For the invariant distribution $\phi(x)$, the flux of the rescaled Fokker-Planck equation is zero, that is:

\begin{equation*}
-\Big(\left[\left(\Theta_X + \Theta_Y\right)x - \Theta_X\right]x\left(1-x\right)-\left(N_X + N_Y + A\right)x + Ai + N_X\Big)\phi(x) + \frac{d}{dx}\left(\phi(x)x\left(1-x\right)\right)= 0
\end{equation*}

With $v(x) = \phi(x)x\left(1-x\right)$, we have:

\begin{equation*}
v'(x) = \left(\left[\left(\Theta_X + \Theta_Y\right)x - \Theta_X\right] + \frac{N_X}{x} - \frac{N_Y}{1-x} + \frac{A(i-x)}{x(1-x)}\right)v(x)
\end{equation*}

and hence

\begin{equation*}
v(x) = Ce^{\frac{1}{2}\left(\Theta_X + \Theta_Y\right)x^2 - \Theta_X x}x^{N_X + Ai}\left(1-x\right)^{N_Y+A - Ai}
\end{equation*}

Respectively,

\begin{equation*}
\phi(x) = Ce^{\frac{1}{2}\left(\Theta_X + \Theta_Y\right)x^2 - \Theta_X x}x^{N_X + Ai - 1}\left(1-x\right)^{N_Y+A - Ai - 1}
\end{equation*}
\end{document}