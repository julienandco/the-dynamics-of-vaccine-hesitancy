\documentclass[10pt,a4paper]{article}
\usepackage[english]{babel}
\usepackage{amsmath}
\delimitershortfall=-1pt
\usepackage{amsthm}
\usepackage{amssymb}
\usepackage{amsfonts}
\usepackage{graphicx}
\usepackage{mathtools}
\usepackage[utf8]{inputenc}
\usepackage{csquotes}
\usepackage[hidelinks, final]{hyperref}
\usepackage[printonlyused]{acronym}
\usepackage{color}
\usepackage{transparent}
\newtheorem{prop}{Proposition}[section]
\begin{document}
	We take a look back at the Fokker-Planck equation for which we computed the Kramers-Moyal expansion before. %TODO: reference
	There, the scaling was $n_i = N_i/N$ and $\theta_i$ constant in N. Afterwards, we will use a slightly different scaling which we will prefer for the Dirichlet limit. We recall the transition rates:
	\begin{align}\label{def:abbreviation_scaled_transition_rates}
	\begin{split}
	f_+(x) &\coloneqq \frac{\theta_X (x+ n_X+ ai)}{\theta_X (x + n_X + ai) + (1-x + n_Y + a(1-i))}\\
	f_-(x) &\coloneqq \frac{\theta_Y(1-x+n_Y+a(1-i))}{(x+n_X+ai) + \theta_Y(1-x+n_Y+a(1-i))}
	\end{split}
	\end{align}
	
	Thus, the limiting Fokker-Planck equation reads:
	\begin{align}
	\begin{split}
		\partial_t \left(u\left(x, t\right)\right) &= -\partial_x\big(\mu u(x,t)\left((1-x)f_+(x) - xf_-(x)\right)\big) \\
		&\qquad + \frac{1}{2N} \partial^2_x\left(\mu u(x,t)\left((1-x)f_+(x) + xf_-(x)\right)\right)
	\end{split}
	\end{align}
	
	\input{weak_eff_lim_v5.tex}
\end{document}