\documentclass[10pt,a4paper]{article}
\usepackage[english]{babel}
\usepackage{amsmath}
\delimitershortfall=-1pt
\usepackage{amsthm}
\usepackage{amssymb}
\usepackage{amsfonts}
\usepackage{graphicx}
\usepackage{mathtools}
\usepackage[utf8]{inputenc}
\usepackage{csquotes}
\usepackage[hidelinks, final]{hyperref}
\usepackage[printonlyused]{acronym}
\usepackage{color}
\usepackage{transparent}
\newtheorem{prop}{Proposition}[section]
\begin{document}
	We take a look back at the Fokker-Planck equation which we computed via the Kramers-Moyal expansion before. %TODO: reference
	There, the scaling was $n_i = N_i/N$ and $\theta_i$ constant in N. Afterwards, we will use a slightly different scaling which we will prefer for the Dirichlet limit. We recall the transition rates:
	\begin{align}\label{def:abbreviation_scaled_transition_rates}
	\begin{split}
	f_+(x) &\coloneqq \frac{\theta_X (x+ n_X+ ai)}{\theta_X (x + n_X + ai) + (1-x + n_Y + a(1-i))}\\
	f_-(x) &\coloneqq \frac{\theta_Y(1-x+n_Y+a(1-i))}{(x+n_X+ai) + \theta_Y(1-x+n_Y+a(1-i))}
	\end{split}
	\end{align}
	
	Thus, the limiting Fokker-Planck equation reads:
	\begin{equation}
	\partial_t \left(u\left(x, t\right)\right) = -\partial_x\big(\mu u(x,t)\left((1-x)f_+(x) - xf_-(x)\right)\big) + \frac{1}{2N} \partial^2_x\left(\mu u(x,t)\left((1-x)f_+(x) + xf_-(x)\right)\right)
	\end{equation}
	
	By rewriting drift and noise term with $n_i = N_i/N, \theta_i = 1- \frac{\Theta_i}{N}$, $h = 1/N$ and neglecting terms of $\mathcal{O}(N^{-2})$ we find (using maxima):

\begin{align*}
	\dagger &\coloneqq -\Big[\frac{(\Theta_X + \Theta_Y)x^3 + ((2\Theta_X + 2\Theta_Y)ai -(\Theta_X + \Theta_Y)a - \Theta_Y - 2\ThetaX)x^2}{(a+1)^2}\Big.\\
	&\qquad + \Big.\frac{((\Theta_X + \Theta_Y)a^2 i^2 + (- (\Theta_X + \Theta_Y)a^2 -(3\Theta_X + \Theta_Y)a)i + \Theta_X a + \Theta_X + N_X + N_Y)x}{(a+1)^2}\Big.\\
	&\qquad - \Big.\frac{\Theta_X a^2 i^2 + (\Theta_X a^2 + (\Theta_X + N_X + N_Y) a)i - (N_X a + N_X)}{(a+1)^2}\Big],\\
\end{align*}

and further: 
\begin{align*}
& \mu(1-x)f_+(x) - \mu xf_-(x)\\
= & \mu(1-x) \frac{(1-h\Theta_X) (x+ hN_X+ ai)}{(1-h\Theta_X) (x + hN_X + ai) + (1-x + hN_Y + a(1-i))}\\
&\qquad - \mu x \frac{(1-h\Theta_Y)(1-x+hN_Y+a(1-i))}{(x+h_NX+ai) + (1-h\Theta_Y)(1-x+hN_Y+a(1-i))}\\
=& \frac{a(i-x)}{a+1} + \mu h \dagger + \mathcal{O}(h^2),\\
\end{align*}

while 

\begin{equation*}
h\left[\mu(1-x)f_+(x) + \mu xf_-(x)\right] = \mu h \frac{ai - 2x^2 - (2ai - a -2)x}{a+1}.
\end{equation*}

Note that:
\begin{align*}
\ddagger &\coloneqq \Big(x+\frac{\sqrt{4a^2i^2-4a^2i+a^2+4a+4}+2ai-a-2}{4}\Big) \times\\
&\qquad \Big(x-\frac{\sqrt{4a^2i^2-4a^2i+a^2+4a+4}-2ai+a+2}{4}\Big)\\
&=\frac{ai - 2x^2 - (2ai - a -2)x}{a+1}
\end{align*}


After rescaling time: $T = \mu h t$, the Fokker-Planck equation becomes

\begin{align*}
\partial_T \left(u\left(x, T\right)\right) = -\partial_x\big(u(x,T)\left(\frac{a(i-x)}{\mu h(a+1)} + \dagger \right)\big) + \frac{1}{2} \partial^2_x\left(u(x,T)\frac{ai - 2x^2 - (2ai - a -2)x}{a+1}\right)
\end{align*}

\textcolor{red}{Hier entsteht mein Problem, dass $\mu h$ in den Nenner des Order-0-Terms kommt...}\newline
For the invariant distribution $\phi(x)$, the flux of the rescaled Fokker-Planck equation is zero, that is:

\begin{equation*}
-\Big(\frac{a(i-x)}{\mu h(a+1)} + \dagger\Big)\phi(x) + \frac{1}{2} \frac{d}{dx}\Big(\phi(x)\frac{ai - 2x^2 - (2ai - a -2)x}{a+1}\Big)= 0
\end{equation*}

With $v(x) = \phi(x)\frac{ai - 2x^2 - (2ai - a -2)x}{a+1}$, we have:

\begin{equation*}
v'(x) = -2\Big(\frac{a(i-x)}{\mu h(a+1)} + \dagger\Big)v(x)
\end{equation*}

and hence

\begin{equation*}
v(x) = todo
\end{equation*}

Respectively,

\begin{equation*}
\phi(x) = todo
\end{equation*}


and hence

\begin{equation*}
	v(x) = todo
\end{equation*}

Respectively,

\begin{equation*}
	\phi(x) = todo
\end{equation*}
\end{document}