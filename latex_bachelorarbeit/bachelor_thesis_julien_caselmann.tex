\documentclass[12pt,a4paper,twoside]{article}
\usepackage[english]{babel}
\usepackage{amsmath}
\usepackage{amssymb}
\usepackage{amsfonts}
\usepackage{graphicx}
\usepackage[utf8]{inputenc}
\usepackage[hidelinks, final]{hyperref}
\usepackage[printonlyused]{acronym}
\usepackage{color}
\usepackage{transparent}
\graphicspath{{img/}}
\usepackage[%
backend=biber,
url=false,
style=numeric,
sorting = none,
maxnames=4,
minnames=3,
maxbibnames=99,
giveninits,
uniquename=init]{biblatex}
\addbibresource{bachelor_thesis_julien_caselmann.bib}

\setlength{\voffset}{-28.4mm}
\setlength{\hoffset}{-1in}
\setlength{\topmargin}{20mm}
\setlength{\oddsidemargin}{25mm}
\setlength{\evensidemargin}{25mm}
\setlength{\textwidth}{160mm}

\setlength{\parindent}{0pt}

\setlength{\textheight}{235mm}
\setlength{\footskip}{20mm}
\setlength{\headsep}{50pt}
\setlength{\headheight}{0pt}


\begin{document}
\pagestyle{empty}
%%%% Title page
\begin{titlepage}
\begin{center}
\includegraphics{img/TUMlblack.png}\\[3mm]
\sf
{\Large
  Technische Universit\"at M\"unchen\\[5mm]
  Department of Mathematics\\[8mm]
}
\normalsize
\includegraphics{img/TUMlMblack.png}\\[15mm]

Bachelor's Thesis\\[15mm]

{\Huge
  The Dynamics of Vaccination Hesitancy
}
\bigskip

\normalsize

Julien Caselmann
\end{center}
\vspace*{75mm}

Supervisor: Prof. Dr. Johannes M\"uller
\medskip

Advisor: Prof. Dr. Johannes M\"uller
\medskip

Submission Date: July $1^{\textnormal{st}}$, 2020

\end{titlepage}
%%%% The following has to be signed by hand!

\vspace*{150mm}

I assure the single handed composition of this bachelor's thesis only supported by declared resources.
\bigskip

Munich,
\newpage
%%%% Summary in german
\section*{Zusammenfassung}
Bei einer in englischer Sprache verfassten Arbeit muss eine Zusammenfassung in deutscher Sprache vorangestellt werden.
Daf\"ur ist hier Platz.

\newpage
\tableofcontents
\newpage

%%%% Page numbering restarts here
\pagenumbering{arabic}
\pagestyle{headings}

\newpage

%if you do not use this, delete following packages in preamble:
%acronym

\section*{Abbreviations}
\begin{acronym}[KPMG]%write longest acronym here for nice formatting purposes
	%to use them write \ac{KPMG} or \acp{KPMG for plural in the text}
	\acro{KPMG}{KPMGeneral}
	\acro{SIR}{Susceptibles, Infecteds, Recovereds}
	%if one acronym's plural form is more than just an "s" attached, you have to specify that:
	\acro{SQ}{sexy query}
	\acroplural{SQ}[SQs]{sexy queries}
\end{acronym}

\section{Introduction}
%How to input a inkscape image:
%\begin{figure}
%	\centering %(or not)
%	\def\svgwidth{175pt}
%	\input{foo/bar/file.pdf_tex}
%	\caption{example}
%	\label{fig:example}
%\end{figure}

\section{Mathematical Tools}

\section{Hesitancy Model}
Firstly, we will have to take a look at a way to model vaccination hesitancy and try to understand, how the ``vacinators" behave and where the dependencies of their behaviour lie. The medical/biological approach over the past few years has inter alia been to compose literature reviews and to form working groups such as in \cite{MacDonald2015} to be able to define so-called ``vaccine hesitancy determinants". Those are categories of reasons, for which individuals would decide not to take a vaccine. Then, a mathematical model of the dynamics had to be found, which takes into account the found determinants and one can analyse mathematically to potentially define explicit influence factors for the vaccine hesitancy dynamics. 

Over time, several different models have been elaborated, but only one will be presented in this thesis: the significant work of Bauch and Bhattacharyya in \cite{Bauch2012}. They developed a behaviour-incidence model with the aim to describe the dynamics of the number of vaccinators $x$ depending on different factors such as vaccine risk, vaccine efficacy and the number of infectious persons. The model reads as follows:
\begin{align}\label{model:bauch}
	\begin{split}
	\frac{dS}{dt} &= \mu\left(1-\epsilon x\right) - \mu S - \beta SI - \tau S\\
	&\\
	\frac{dI}{dt} &= -\mu I + \beta SI - \gamma I + \tau S\\
	&\\
	\frac{dx}{dt} &= \kappa x\left(1 - x\right)\left(I - \omega\right)
	\end{split}
\end{align}
The variables $S$ and $I$ follow the convention from the well-studied \ac{SIR} model developed by Kermack and McKendrick in 1927 (for more details, see \cite{Muller2015}) and $x$ represents the relative number of vaccinators in the observed population. A lot of other different variables occur in \eqref{model:bauch}, so they will be listed here to get a better overview:
\begin{align}\label{table:bauch_model_variables}
	\begin{split}
	\mu &: \textnormal{birth and death rate of the population}\\
	\epsilon &: \textnormal{vaccine efficacy}\\
	\beta &: \textnormal{infection rate}\\
	\tau &: \textnormal{case importation rate}\\
	\gamma &: \textnormal{recovery rate}\\
	\kappa &: \textnormal{scale factor}\\
	\omega &: \textnormal{vaccine penalty}
	\end{split}
\end{align}
The vaccine penalty $\omega$ basically describes the amount of risk a vaccination brings.

The modeling idea is that every person starts in the group of susceptibles $S$ and either stays in $S$, moves to the group of infecteds $I$, or gets the vaccine and immediately jumps into the group of recovereds $R$. As the main interest of this model is the dynamics of the vaccinators $x$ and once an individual gets into $R$, a return to $S$ or $I$ is impossible, it is assumed that the recovered have left the model, thus are not mentioned in \eqref{model:bauch}. The dynamics of $S$ are pretty straight-forward: the only way to get into $S$, is to get born into the population, so the only positive term is the birth rate $\mu$. On the other hand, there are multiple ways to get ot of $S$: an individual might either die $\left(-\mu S\right)$, get infected by one individual of $I$ $\left(-\beta SI\right)$, be born as a child of vaccinators and get vaccined successfully at birth $\left(-\mu\epsilon x\right)$ or come from another population as an infected and immediately jump from $S$ to $I$ $\left(-\tau S\right)$. 

$I$ is designed in a similar way: newly infecteds were either infected in the observed population $\left(+\beta SI\right)$ or came from another population and brougth the disease with them $\left(+\tau S\right)$. Again, an individual leaves $I$ at death $\left(-\mu I\right)$ or after recovering from the disease $\left(-\gamma I\right)$. 

Finally, let us take a look at the vaccinator dynamics%TODO: what exactly does the scale factor kappa do? what would be an example of it?
. They consist of two factors. The first one simply puts the relative number of vaccinators $x$ into relation with the non-vaccinators $\left(1-x\right)$, the second is the interesting one! It computes the difference between $I$ and the risk $\omega$ of taking the vaccine. It basically depicts the danger emerging from the disease, perceived by the population. When there is a high amount of infecteds and the vaccine penalty is low, the difference will be positive and quite big, leading to a higher growth rate of $x$ (more people will become vaccinators and vaccine their children at birth) and vice versa. When $\omega$ and $I$ are approximatively the same, the growth/decrease of $x$ will turn out quite insignificant.

One may notice that the dynamics of the vaccinators $x$ as presented in the model \eqref{model:bauch} are quite basic and not realistic at all. They imply that the only influencing factor in the switch between vaccinator and non-vaccinator is this kind of ``payoff comparison" described by the difference $I - \omega$ and attribute it an extreme power. If for instance,we considered a population with $N$ individuals, almost no vaccinators at a time $t$, say $x(t) \leq 0.05$, a massive amount of infecteds $I \sim N$ and a somewhat decent vaccine risk. Even if the risk was still high, the enormous $I$ would lead to the ``payoff" being quite big. According to this model, a tremendous amount of anti-vaccinators would suddenly become vaccinators, despite their ideologies, fears and maybe also religious beliefs, whereas experience tells us the exact opposite. Even in such dramatic context, the path from being a staunch anti-vaccinator to taking a flu shot into consideration can be tedious, even if the vaccine is allegedly safe or herd immunity for the whole population could be reached with that shot \cite{Bednarz2020, Health2019}. Additionally, exchange between those two camps or differently weighted opinions are completely excluded from this model, yet highly present in the real world.

\section{Opinion model}
blablabla für opinion modeling gibt es verschiedene approaches :2-3 Quellen. ein ansatz von deGroot, der sich mehr auf konsens finden fokussiert, der andere ansatz von friedkin-johnsen opinion dynamics model, dann erweitertes f-J von chitra, bounded-confidence model von hegselmann-krause. Hier beschäftigen wir uns mit moran model. Funfact: eig für populationsgenetik entwickelt, aber isch dasselbe.

Am besten: die MOdelle die erwähnt werden evtl ein paar ganz wilde, aber überwiegend die, die mich dann zu meinem baba modell inspiriert haben
\subsection{Voter model}
bums knyckig definieren
\subsection{Noisy voter model}
oben nicht ganz realistisch, also erweitern wir: meinungswechsel nur mit prob und nicht konst. rate.
moran model erweitern, wsl nicht mal eigene subsection nötig, nur gemacht um zu strukturieren beim gehirnstürmen
\subsection{Filter bubbles: zealot model}
aber: obiges ist auch schwul, da on se long run nur eine meinung überlebt $\rightarrow$ unrealistisch. außerdem ist es auch nicht warh, dass jeder so reif ist, regelmäßig über seine ansichten zu reflektieren, geschweige denn sie zu überdenken $\rightarrow$ führen fantiker ein, die nur seeehr langsam/gar nicht meinung wechseln (zeloten).
\subsection{Sano model}
haut sich des? %TODO: do more research on sano model, so I know whether imma use it or nah
\section{My Model}
\section{Discussion}

\newpage
\printbibliography
\end{document}



